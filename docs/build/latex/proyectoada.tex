%% Generated by Sphinx.
\def\sphinxdocclass{report}
\documentclass[letterpaper,10pt,spanish]{sphinxmanual}
\ifdefined\pdfpxdimen
   \let\sphinxpxdimen\pdfpxdimen\else\newdimen\sphinxpxdimen
\fi \sphinxpxdimen=.75bp\relax
\ifdefined\pdfimageresolution
    \pdfimageresolution= \numexpr \dimexpr1in\relax/\sphinxpxdimen\relax
\fi
%% let collapsible pdf bookmarks panel have high depth per default
\PassOptionsToPackage{bookmarksdepth=5}{hyperref}

\PassOptionsToPackage{booktabs}{sphinx}
\PassOptionsToPackage{colorrows}{sphinx}

\PassOptionsToPackage{warn}{textcomp}
\usepackage[utf8]{inputenc}
\ifdefined\DeclareUnicodeCharacter
% support both utf8 and utf8x syntaxes
  \ifdefined\DeclareUnicodeCharacterAsOptional
    \def\sphinxDUC#1{\DeclareUnicodeCharacter{"#1}}
  \else
    \let\sphinxDUC\DeclareUnicodeCharacter
  \fi
  \sphinxDUC{00A0}{\nobreakspace}
  \sphinxDUC{2500}{\sphinxunichar{2500}}
  \sphinxDUC{2502}{\sphinxunichar{2502}}
  \sphinxDUC{2514}{\sphinxunichar{2514}}
  \sphinxDUC{251C}{\sphinxunichar{251C}}
  \sphinxDUC{2572}{\textbackslash}
\fi
\usepackage{cmap}
\usepackage[T1]{fontenc}
\usepackage{amsmath,amssymb,amstext}
\usepackage{babel}



\usepackage{tgtermes}
\usepackage{tgheros}
\renewcommand{\ttdefault}{txtt}



\usepackage[Sonny]{fncychap}
\ChNameVar{\Large\normalfont\sffamily}
\ChTitleVar{\Large\normalfont\sffamily}
\usepackage{sphinx}

\fvset{fontsize=auto}
\usepackage{geometry}


% Include hyperref last.
\usepackage{hyperref}
% Fix anchor placement for figures with captions.
\usepackage{hypcap}% it must be loaded after hyperref.
% Set up styles of URL: it should be placed after hyperref.
\urlstyle{same}

\addto\captionsspanish{\renewcommand{\contentsname}{Contents:}}

\usepackage{sphinxmessages}
\setcounter{tocdepth}{1}



\title{PROYECTO ADA}
\date{17 de diciembre de 2023}
\release{1.0}
\author{Jandi Ramirez, Paula Lemus, Camilo Viedma}
\newcommand{\sphinxlogo}{\vbox{}}
\renewcommand{\releasename}{Versión}
\makeindex
\begin{document}

\ifdefined\shorthandoff
  \ifnum\catcode`\=\string=\active\shorthandoff{=}\fi
  \ifnum\catcode`\"=\active\shorthandoff{"}\fi
\fi

\pagestyle{empty}
\sphinxmaketitle
\pagestyle{plain}
\sphinxtableofcontents
\pagestyle{normal}
\phantomsection\label{\detokenize{index::doc}}


\sphinxstepscope


\chapter{PROYECTO ADA}
\label{\detokenize{modules:proyecto-ada}}\label{\detokenize{modules::doc}}
\sphinxstepscope


\section{Funcionales package}
\label{\detokenize{Funcionales:funcionales-package}}\label{\detokenize{Funcionales::doc}}

\subsection{Submodules}
\label{\detokenize{Funcionales:submodules}}

\subsection{Funcionales.calendario module}
\label{\detokenize{Funcionales:module-Funcionales.calendario}}\label{\detokenize{Funcionales:funcionales-calendario-module}}\index{module@\spxentry{module}!Funcionales.calendario@\spxentry{Funcionales.calendario}}\index{Funcionales.calendario@\spxentry{Funcionales.calendario}!module@\spxentry{module}}\index{elegir\_archivo() (en el módulo Funcionales.calendario)@\spxentry{elegir\_archivo()}\spxextra{en el módulo Funcionales.calendario}}

\begin{fulllineitems}
\phantomsection\label{\detokenize{Funcionales:Funcionales.calendario.elegir_archivo}}
\pysigstartsignatures
\pysiglinewithargsret{\sphinxcode{\sphinxupquote{Funcionales.calendario.}}\sphinxbfcode{\sphinxupquote{elegir\_archivo}}}{}{}
\pysigstopsignatures
\sphinxAtStartPar
Permite al usuario seleccionar un archivo y genera un calendario a partir de los datos del archivo.
\begin{quote}\begin{description}
\sphinxlineitem{Parámetros}
\sphinxAtStartPar
\sphinxstyleliteralstrong{\sphinxupquote{ruta}} (\sphinxstyleliteralemphasis{\sphinxupquote{str}}) \textendash{} La ruta del archivo.

\sphinxlineitem{Devuelve}
\sphinxAtStartPar
El calendario generado.

\sphinxlineitem{Tipo del valor devuelto}
\sphinxAtStartPar
list

\end{description}\end{quote}

\end{fulllineitems}

\index{generar\_calendario() (en el módulo Funcionales.calendario)@\spxentry{generar\_calendario()}\spxextra{en el módulo Funcionales.calendario}}

\begin{fulllineitems}
\phantomsection\label{\detokenize{Funcionales:Funcionales.calendario.generar_calendario}}
\pysigstartsignatures
\pysiglinewithargsret{\sphinxcode{\sphinxupquote{Funcionales.calendario.}}\sphinxbfcode{\sphinxupquote{generar\_calendario}}}{\sphinxparam{\DUrole{n}{n\_equipos}}\sphinxparamcomma \sphinxparam{\DUrole{n}{min\_partidos}}\sphinxparamcomma \sphinxparam{\DUrole{n}{max\_partidos}}\sphinxparamcomma \sphinxparam{\DUrole{n}{distancias}}}{}
\pysigstopsignatures
\sphinxAtStartPar
Genera un calendario para un torneo de equipos.
\begin{quote}\begin{description}
\sphinxlineitem{Parámetros}\begin{itemize}
\item {} 
\sphinxAtStartPar
\sphinxstyleliteralstrong{\sphinxupquote{n\_equipos}} (\sphinxstyleliteralemphasis{\sphinxupquote{int}}) \textendash{} El número de equipos en el torneo.

\item {} 
\sphinxAtStartPar
\sphinxstyleliteralstrong{\sphinxupquote{min\_partidos}} (\sphinxstyleliteralemphasis{\sphinxupquote{int}}) \textendash{} El número mínimo de partidos que un equipo debe jugar.

\item {} 
\sphinxAtStartPar
\sphinxstyleliteralstrong{\sphinxupquote{max\_partidos}} (\sphinxstyleliteralemphasis{\sphinxupquote{int}}) \textendash{} El número máximo de partidos que un equipo puede jugar.

\item {} 
\sphinxAtStartPar
\sphinxstyleliteralstrong{\sphinxupquote{distancias}} (\sphinxstyleliteralemphasis{\sphinxupquote{list}}) \textendash{} Una matriz de distancias entre los equipos.

\end{itemize}

\sphinxlineitem{Devuelve}
\sphinxAtStartPar
El calendario del torneo.

\sphinxlineitem{Tipo del valor devuelto}
\sphinxAtStartPar
list

\end{description}\end{quote}

\end{fulllineitems}

\index{generar\_calendario\_desde\_archivo() (en el módulo Funcionales.calendario)@\spxentry{generar\_calendario\_desde\_archivo()}\spxextra{en el módulo Funcionales.calendario}}

\begin{fulllineitems}
\phantomsection\label{\detokenize{Funcionales:Funcionales.calendario.generar_calendario_desde_archivo}}
\pysigstartsignatures
\pysiglinewithargsret{\sphinxcode{\sphinxupquote{Funcionales.calendario.}}\sphinxbfcode{\sphinxupquote{generar\_calendario\_desde\_archivo}}}{\sphinxparam{\DUrole{n}{ruta}}}{}
\pysigstopsignatures
\sphinxAtStartPar
Genera un calendario a partir de los datos de un archivo.
\begin{quote}\begin{description}
\sphinxlineitem{Parámetros}
\sphinxAtStartPar
\sphinxstyleliteralstrong{\sphinxupquote{ruta}} (\sphinxstyleliteralemphasis{\sphinxupquote{str}}) \textendash{} La ruta del archivo.

\sphinxlineitem{Devuelve}
\sphinxAtStartPar
El calendario generado.

\sphinxlineitem{Tipo del valor devuelto}
\sphinxAtStartPar
list

\end{description}\end{quote}

\end{fulllineitems}

\index{imprimir\_calendario() (en el módulo Funcionales.calendario)@\spxentry{imprimir\_calendario()}\spxextra{en el módulo Funcionales.calendario}}

\begin{fulllineitems}
\phantomsection\label{\detokenize{Funcionales:Funcionales.calendario.imprimir_calendario}}
\pysigstartsignatures
\pysiglinewithargsret{\sphinxcode{\sphinxupquote{Funcionales.calendario.}}\sphinxbfcode{\sphinxupquote{imprimir\_calendario}}}{\sphinxparam{\DUrole{n}{calendario}}\sphinxparamcomma \sphinxparam{\DUrole{n}{n\_equipos}}\sphinxparamcomma \sphinxparam{\DUrole{n}{min\_partidos}}\sphinxparamcomma \sphinxparam{\DUrole{n}{max\_partidos}}}{}
\pysigstopsignatures
\sphinxAtStartPar
Imprime el calendario del torneo.
\begin{quote}\begin{description}
\sphinxlineitem{Parámetros}\begin{itemize}
\item {} 
\sphinxAtStartPar
\sphinxstyleliteralstrong{\sphinxupquote{calendario}} (\sphinxstyleliteralemphasis{\sphinxupquote{list}}) \textendash{} El calendario del torneo.

\item {} 
\sphinxAtStartPar
\sphinxstyleliteralstrong{\sphinxupquote{n\_equipos}} (\sphinxstyleliteralemphasis{\sphinxupquote{int}}) \textendash{} El número de equipos en el torneo.

\item {} 
\sphinxAtStartPar
\sphinxstyleliteralstrong{\sphinxupquote{min\_partidos}} (\sphinxstyleliteralemphasis{\sphinxupquote{int}}) \textendash{} El número mínimo de partidos que un equipo debe jugar.

\item {} 
\sphinxAtStartPar
\sphinxstyleliteralstrong{\sphinxupquote{max\_partidos}} (\sphinxstyleliteralemphasis{\sphinxupquote{int}}) \textendash{} El número máximo de partidos que un equipo puede jugar.

\end{itemize}

\end{description}\end{quote}

\end{fulllineitems}

\index{leer\_datos\_desde\_archivo() (en el módulo Funcionales.calendario)@\spxentry{leer\_datos\_desde\_archivo()}\spxextra{en el módulo Funcionales.calendario}}

\begin{fulllineitems}
\phantomsection\label{\detokenize{Funcionales:Funcionales.calendario.leer_datos_desde_archivo}}
\pysigstartsignatures
\pysiglinewithargsret{\sphinxcode{\sphinxupquote{Funcionales.calendario.}}\sphinxbfcode{\sphinxupquote{leer\_datos\_desde\_archivo}}}{\sphinxparam{\DUrole{n}{ruta}}}{}
\pysigstopsignatures
\sphinxAtStartPar
Lee los datos de un archivo.
\begin{quote}\begin{description}
\sphinxlineitem{Parámetros}
\sphinxAtStartPar
\sphinxstyleliteralstrong{\sphinxupquote{ruta}} (\sphinxstyleliteralemphasis{\sphinxupquote{str}}) \textendash{} La ruta del archivo.

\sphinxlineitem{Devuelve}
\sphinxAtStartPar
Los datos leídos del archivo.

\sphinxlineitem{Tipo del valor devuelto}
\sphinxAtStartPar
tuple

\end{description}\end{quote}

\end{fulllineitems}



\subsection{Module contents}
\label{\detokenize{Funcionales:module-Funcionales}}\label{\detokenize{Funcionales:module-contents}}\index{module@\spxentry{module}!Funcionales@\spxentry{Funcionales}}\index{Funcionales@\spxentry{Funcionales}!module@\spxentry{module}}

\chapter{Indices and tables}
\label{\detokenize{index:indices-and-tables}}\begin{itemize}
\item {} 
\sphinxAtStartPar
\DUrole{xref,std,std-ref}{genindex}

\item {} 
\sphinxAtStartPar
\DUrole{xref,std,std-ref}{modindex}

\item {} 
\sphinxAtStartPar
\DUrole{xref,std,std-ref}{search}

\end{itemize}


\renewcommand{\indexname}{Índice de Módulos Python}
\begin{sphinxtheindex}
\let\bigletter\sphinxstyleindexlettergroup
\bigletter{f}
\item\relax\sphinxstyleindexentry{Funcionales}\sphinxstyleindexpageref{Funcionales:\detokenize{module-Funcionales}}
\item\relax\sphinxstyleindexentry{Funcionales.calendario}\sphinxstyleindexpageref{Funcionales:\detokenize{module-Funcionales.calendario}}
\end{sphinxtheindex}

\renewcommand{\indexname}{Índice}
\printindex
\end{document}